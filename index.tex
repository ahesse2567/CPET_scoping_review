% Options for packages loaded elsewhere
\PassOptionsToPackage{unicode}{hyperref}
\PassOptionsToPackage{hyphens}{url}
\PassOptionsToPackage{dvipsnames,svgnames,x11names}{xcolor}
%
\documentclass[
  letterpaper,
  DIV=11,
  numbers=noendperiod]{scrartcl}

\usepackage{amsmath,amssymb}
\usepackage{iftex}
\ifPDFTeX
  \usepackage[T1]{fontenc}
  \usepackage[utf8]{inputenc}
  \usepackage{textcomp} % provide euro and other symbols
\else % if luatex or xetex
  \usepackage{unicode-math}
  \defaultfontfeatures{Scale=MatchLowercase}
  \defaultfontfeatures[\rmfamily]{Ligatures=TeX,Scale=1}
\fi
\usepackage{lmodern}
\ifPDFTeX\else  
    % xetex/luatex font selection
\fi
% Use upquote if available, for straight quotes in verbatim environments
\IfFileExists{upquote.sty}{\usepackage{upquote}}{}
\IfFileExists{microtype.sty}{% use microtype if available
  \usepackage[]{microtype}
  \UseMicrotypeSet[protrusion]{basicmath} % disable protrusion for tt fonts
}{}
\makeatletter
\@ifundefined{KOMAClassName}{% if non-KOMA class
  \IfFileExists{parskip.sty}{%
    \usepackage{parskip}
  }{% else
    \setlength{\parindent}{0pt}
    \setlength{\parskip}{6pt plus 2pt minus 1pt}}
}{% if KOMA class
  \KOMAoptions{parskip=half}}
\makeatother
\usepackage{xcolor}
\setlength{\emergencystretch}{3em} % prevent overfull lines
\setcounter{secnumdepth}{5}
% Make \paragraph and \subparagraph free-standing
\makeatletter
\ifx\paragraph\undefined\else
  \let\oldparagraph\paragraph
  \renewcommand{\paragraph}{
    \@ifstar
      \xxxParagraphStar
      \xxxParagraphNoStar
  }
  \newcommand{\xxxParagraphStar}[1]{\oldparagraph*{#1}\mbox{}}
  \newcommand{\xxxParagraphNoStar}[1]{\oldparagraph{#1}\mbox{}}
\fi
\ifx\subparagraph\undefined\else
  \let\oldsubparagraph\subparagraph
  \renewcommand{\subparagraph}{
    \@ifstar
      \xxxSubParagraphStar
      \xxxSubParagraphNoStar
  }
  \newcommand{\xxxSubParagraphStar}[1]{\oldsubparagraph*{#1}\mbox{}}
  \newcommand{\xxxSubParagraphNoStar}[1]{\oldsubparagraph{#1}\mbox{}}
\fi
\makeatother


\providecommand{\tightlist}{%
  \setlength{\itemsep}{0pt}\setlength{\parskip}{0pt}}\usepackage{longtable,booktabs,array}
\usepackage{calc} % for calculating minipage widths
% Correct order of tables after \paragraph or \subparagraph
\usepackage{etoolbox}
\makeatletter
\patchcmd\longtable{\par}{\if@noskipsec\mbox{}\fi\par}{}{}
\makeatother
% Allow footnotes in longtable head/foot
\IfFileExists{footnotehyper.sty}{\usepackage{footnotehyper}}{\usepackage{footnote}}
\makesavenoteenv{longtable}
\usepackage{graphicx}
\makeatletter
\def\maxwidth{\ifdim\Gin@nat@width>\linewidth\linewidth\else\Gin@nat@width\fi}
\def\maxheight{\ifdim\Gin@nat@height>\textheight\textheight\else\Gin@nat@height\fi}
\makeatother
% Scale images if necessary, so that they will not overflow the page
% margins by default, and it is still possible to overwrite the defaults
% using explicit options in \includegraphics[width, height, ...]{}
\setkeys{Gin}{width=\maxwidth,height=\maxheight,keepaspectratio}
% Set default figure placement to htbp
\makeatletter
\def\fps@figure{htbp}
\makeatother
% definitions for citeproc citations
\NewDocumentCommand\citeproctext{}{}
\NewDocumentCommand\citeproc{mm}{%
  \begingroup\def\citeproctext{#2}\cite{#1}\endgroup}
\makeatletter
 % allow citations to break across lines
 \let\@cite@ofmt\@firstofone
 % avoid brackets around text for \cite:
 \def\@biblabel#1{}
 \def\@cite#1#2{{#1\if@tempswa , #2\fi}}
\makeatother
\newlength{\cslhangindent}
\setlength{\cslhangindent}{1.5em}
\newlength{\csllabelwidth}
\setlength{\csllabelwidth}{3em}
\newenvironment{CSLReferences}[2] % #1 hanging-indent, #2 entry-spacing
 {\begin{list}{}{%
  \setlength{\itemindent}{0pt}
  \setlength{\leftmargin}{0pt}
  \setlength{\parsep}{0pt}
  % turn on hanging indent if param 1 is 1
  \ifodd #1
   \setlength{\leftmargin}{\cslhangindent}
   \setlength{\itemindent}{-1\cslhangindent}
  \fi
  % set entry spacing
  \setlength{\itemsep}{#2\baselineskip}}}
 {\end{list}}
\usepackage{calc}
\newcommand{\CSLBlock}[1]{\hfill\break\parbox[t]{\linewidth}{\strut\ignorespaces#1\strut}}
\newcommand{\CSLLeftMargin}[1]{\parbox[t]{\csllabelwidth}{\strut#1\strut}}
\newcommand{\CSLRightInline}[1]{\parbox[t]{\linewidth - \csllabelwidth}{\strut#1\strut}}
\newcommand{\CSLIndent}[1]{\hspace{\cslhangindent}#1}

\KOMAoption{captions}{tableheading}
\makeatletter
\@ifpackageloaded{caption}{}{\usepackage{caption}}
\AtBeginDocument{%
\ifdefined\contentsname
  \renewcommand*\contentsname{Table of contents}
\else
  \newcommand\contentsname{Table of contents}
\fi
\ifdefined\listfigurename
  \renewcommand*\listfigurename{List of Figures}
\else
  \newcommand\listfigurename{List of Figures}
\fi
\ifdefined\listtablename
  \renewcommand*\listtablename{List of Tables}
\else
  \newcommand\listtablename{List of Tables}
\fi
\ifdefined\figurename
  \renewcommand*\figurename{Figure}
\else
  \newcommand\figurename{Figure}
\fi
\ifdefined\tablename
  \renewcommand*\tablename{Table}
\else
  \newcommand\tablename{Table}
\fi
}
\@ifpackageloaded{float}{}{\usepackage{float}}
\floatstyle{ruled}
\@ifundefined{c@chapter}{\newfloat{codelisting}{h}{lop}}{\newfloat{codelisting}{h}{lop}[chapter]}
\floatname{codelisting}{Listing}
\newcommand*\listoflistings{\listof{codelisting}{List of Listings}}
\makeatother
\makeatletter
\makeatother
\makeatletter
\@ifpackageloaded{caption}{}{\usepackage{caption}}
\@ifpackageloaded{subcaption}{}{\usepackage{subcaption}}
\makeatother
\ifLuaTeX
  \usepackage{selnolig}  % disable illegal ligatures
\fi
\usepackage{bookmark}

\IfFileExists{xurl.sty}{\usepackage{xurl}}{} % add URL line breaks if available
\urlstyle{same} % disable monospaced font for URLs
\hypersetup{
  pdftitle={Popularity And Prevalence Of Gas Exchange Data Processing Methods: A Scoping Review},
  pdfkeywords={Data Averaging, Outlier Removal, Interpolation},
  colorlinks=true,
  linkcolor={blue},
  filecolor={Maroon},
  citecolor={Blue},
  urlcolor={Blue},
  pdfcreator={LaTeX via pandoc}}

\title{Popularity And Prevalence Of Gas Exchange Data Processing
Methods: A Scoping Review}
\author{}
\date{}

\begin{document}
\maketitle
\begin{abstract}
Cardiopulmonary exercise testing involves collecting variable
breath-by-breath data, sometimes requiring data processing of outlier
removal, interpolation, and averaging before later analysis. These data
processing choices, such as averaging duration, are known to affect
calculated values such as VO\textsubscript{2}max. However, assessing the
effects of data processing without knowing popular methods worth
comparing is difficult. In addition, such details aid study
reproduction. We conducted a scoping review of articles with exercise
testing that collected data breath-by-breath from three databases. Of
the 8,351 articles, 376 (4.5 ± 0.4\%) and 581 (7.0 ± 0.5\%) described
outlier removal and interpolation, respectively. An estimated 66.8 ±
2.8\% reported averaging methods (n = 1078). Commonly documented outlier
cutoffs were ± 3 or 4 SD (39.1\% and 51.6\%, respectively). The
dominating interpolation duration and procedure were one second (93.9\%)
and linear interpolation (92.5\%). Averaging methods commonly described
were 30 (30.9\%), 60 (12.4\%), 15 (11.6\%), 10 (11.0\%), and 20 (8.1\%)
second bin averages. This shows that studies collecting breath-by-breath
data often lack detailed descriptions of data processing methods,
particularly for outlier removal and interpolation. While averaging
methods are more commonly reported, improved documentation across all
processing methods will enhance reproducibility and facilitate future
research comparing data processing choices.
\end{abstract}

\section{Introduction}\label{introduction}

Clinicians and researchers commonly use cardiopulmonary exercise testing
(CPET) to determine maximal aerobic capacity (VO\textsubscript{2}max),
ventilatory thresholds, and VO\textsubscript{2} kinetics. Such values
help categorize fitness, predict disease risk, and guide exercise {[}1,
p.~162{]}. Using CPET results to guide exercise, especially relative to
thresholds, produces better improvements due to more consistent and
predictable metabolic responses {[}2{]}. Therefore, incorrectly
calculating or identifying these values limits CPET benefits.

Calculating the above values often requires data processing when CPET
data is collected breath-by-breath (BBB) as it is highly variable
{[}3{]}. CPET data processing usually involves outlier removal, optional
interpolation to regular intervals, and averaging to more accurately
reflect whole-body metabolism {[}3{]}. Previous research has shown that
data averaging influences CPET values. Averaging over longer durations
reduces VO\textsubscript{2}max and VO\textsubscript{2} plateau detection
{[}3,4,4--17{]}. We are unaware of research on the effects of data
processing and locating ventilatory thresholds.

Many studies remove outliers by finding points ±3 or ±4 standard
deviations (SD) beyond the local mean (i.e., a prediction interval).
These cutoffs are common because the relatively small sample size of BBB
gas exchange data often contains more values beyond 3 or 4 SD than one
would predict from an assumed Gaussian distribution {[}18{]}. More
outliers appear than expected because of both conscious and unconscious
alterations of breathing patterns, including swallowing and coughing
{[}18{]}. We are unaware of prior research that examines how different
outlier removal strategies affect VO\textsubscript{2}max, ventilatory
thresholds, and VO\textsubscript{2} kinetics.

Interpolation, often to one-second intervals, is common in
VO\textsubscript{2} kinetics research to ``ensemble'' average repeated
transitions to minimize variability {[}18,19{]}. Although this does not
affect parameter estimates, one-second interpolation has been criticized
for artificially narrowing confidence intervals {[}20--23{]}. As before,
we are unaware of research specifically investigating how interpolation
affects VO\textsubscript{2}max and ventilatory threshold identification.

Data processing choices, such as averaging and interpolation, impact
CPET variables or their confidence intervals. Existing surveys {[}3{]}
and studies {[}7{]} are small and focused on averaging methods only,
finding time-based bin averages (e.g., 30-second averages) were popular.
A larger sample can, therefore, better describe how often all data
processing steps are described.

Before conducting this scoping review, we anecdotally observed that many
articles using CPET data did not report all data processing steps,
especially outlier removal and interpolation details. This may hamper
reproduction or replication attempts, which have become a more prominent
issue in science within the past decade {[}24,25{]} Therefore, to assist
with conducting future research on the effects of data processing on
CPET values and to evaluate the methodological reproducibility of
research using BBB gas exchange data generally, we conducted a broad
scoping review to identify the frequency of reporting, and popularity of
outlier removal, interpolation, and data averaging methods.

\section{Methods}\label{sec-methods}

\subsection{Design and Eligibility
Criteria}\label{design-and-eligibility-criteria}

This scoping review surveyed gas exchange data processing choices in
original, peer-reviewed studies, summarizing the reporting frequency and
methods for outlier removal, interpolation, and averaging. It is based
on a dissertation chapter by the first author {[}26{]}. These methods
{[}27{]} and results {[}28{]} are modeled on the PRISMA scoping review
extension guidelines. Eligible articles were original, peer-reviewed
articles, with BBB gas exchange data, human participants, in English,
with a DOI. We imposed no date restriction.

\subsection{Information Sources and
Search}\label{information-sources-and-search}

We acquired data from the Ovid-MEDLINE, Scopus, and Web of Science
databases with the guidance of a university librarian. The electronic
search strategy for the Ovid-MEDLINE database can be found in the
supplemental materials.

Our search output comprised article identifiers like DOIs. To find
missing DOIs, we employed the PubMed Central ID Converter API {[}29{]}
using Python. Full texts were accessed via publisher text and data
mining APIs using Python, \href{https://unpaywall.org/}{unpaywall.org}
using the unpywall Python package, through custom-built web-scraping
scripts, or manually. Our library subscription did not permit access to
1,549 articles.

\subsection{Selection of Sources of
Evidence}\label{selection-of-sources-of-evidence}

This study used a single screening process because it differs from most
scoping reviews. It only requires an exercise test with BBB gas exchange
data collection rather than a more complex assessment of the overall
methodology and intervention.

\subsubsection{Text Analysis and
Screening}\label{text-analysis-and-screening}

Despite database search filters, we screened additional non-English,
non-human, and non-original articles such as reviews, meta-analyses, and
protocol registrations, in addition to case studies. We manually
analyzed a subset of articles to help build machine learning (ML)
classifiers and construct RegExs described below. These ML classifiers
and RegExs helped identify ineligible articles. This computerized
screening required converting full-text PDF and EPUB documents into
plain text files. Plain text files were normalized by transforming text
to lowercase, removing hyphenations and extra whitespace, and correcting
some plain text conversion-induced errors.

Following the normalization, we identified and removed articles that
failed to correctly convert into text format, spotted non-English
articles using the fasttext Python module {[}30{]} and employed a random
forest classifier from the sklearn Python package {[}31{]} to detect
ineligible articles based on our criteria. We manually reviewed
potentially ineligible articles flagged by the ML classifier.

Next, we identified BBB articles using RegExs. Articles were considered
BBB articles if their text contained variations of the phrase
``breath-by-breath'', or if their text included the make or model of a
known BBB analyzer. Breath-by-breath brands and analyzers we included
were Oxycon and Carefusion brands, Medgraphics Ultima, CPX, CCM, and
CardiO\textsubscript{2} models, Sensormedics Encore and 2900 models,
Cosmed quark, k4, and k5 models, and the Minato RM-200, AE-280S,
AE-300S, and AE-310S models. In total, we identified 8,417 articles.

Within this subset, we performed a similar RegEx search for studies that
documented using Douglas Bags or mixing chambers and excluded those
articles. The full details are described in the ``data charting
process'' section.

\subsubsection{Data Charting Process}\label{data-charting-process}

RegExs identified the presence of short phrases likely indicating that
the authors described these methodological details. If present, we
extracted a ``snippet'' of text surrounding those phrases for later
manual analysis by obtaining approximately 200 surrounding characters.
We then recorded the methods from these snippets. In all cases, methods
were only considered documented if the snippets provided at least some
specific information. For example, articles stating outlying breaths
were removed but without describing the outlier criteria were considered
``not described.'' Finally, we read the full-text article to accurately
document the data when snippets were ambiguous.

The data charting subsections below provide text extraction examples.
Extracted texts were normalized to lowercase, with end-of-line
hyphenation and unnecessary white space removed before capitalizing
certain keywords for readability. Therefore, formatting varies and may
include unconventional spacing and Unicode characters.

We analyzed all eligible BBB articles for outlier and interpolation
methods because fewer articles described these methods
(\textasciitilde5\%) and the phrases were more distinct. In contrast, we
analyzed a random subset of articles to document data averaging methods
because far more articles described their averaging methods. Early
estimates as we developed our RegExs were that \textasciitilde60\% or
5,050 articles had some averaging details. Furthermore, the phrases
associated with averaging methods are more generic and often refer to
other study aspects, such as heart rate averaging periods. Given the
large number of articles, we needed a minimum sample size of 1,068 based
on a 95\% confidence interval and a maximum margin of error of ±3\%,
assuming a proportion of 0.5. However, we raised this to 1,100 in
anticipation of finding ineligible articles that eluded our previous
text screening.

\paragraph{Outliers}\label{outliers}

Our outlier RegExs identified phrases like ``swallowing'', ``coughing'',
``errant'', ``aberrant'', and references to the ``local mean,''
``prediction interval,'' or a specific standard deviation limit such as
±3 or ±4. For example, our RegExs found '' errant''; '' local mean'';
and ``breath-by-breath ̇vo2 data from each step transition were
initially edited to exclude errant breaths by removing values lying more
than 4 sd'' from {[}32{]}. We gathered snippets surrounding those
phrases and combined them when overlapping, thus producing

\begin{quote}
\begin{quote}
y{[}hb+mb{]} data (quaresima \& ferrari, 2009). expressed as 2.5 data
analysis and kinetic modelling the breath-by-breath ̇vo2 data from each
step transition were initially edited to exclude errant breaths by
removing values lying more than 4 sd from the local mean determined
using a five-breath rolling\textbackslash n\textbackslash x0c1932 breese
et al.~and deoxy{[}hb+mb{]} responses were subavera
\end{quote}
\end{quote}

We recorded the outlier limit as ±4 SD and the outlier function as a
rolling 5-breath whole mean average.

\paragraph{Interpolation}\label{interpolation}

Nearly all articles describing interpolation methods used variations of
``interpolate.'' The remaining phrases were infrequent and inconsistent
enough that interpolation methods were only described for those articles
when discovered by chance. To illustrate interpolation documentation,
our RegExs extracted the snippet from {[}33{]}.

\begin{quote}
\begin{quote}
the v̇ o2 data from gd and gl exercise bouts were modeled to characterize
the oxygen uptake kinetics following the methods described by bell et
al.~(2001). breath-by-breath v̇ o2 data were linearly INTERPOLATed to
provide second-by-second values. phase 1 data (i.e.~the cardiodynamic
component), from the first ∼20 s of exercise, were omitted from the
kinetics analysis because phase 1 is not directly repres
\end{quote}
\end{quote}

We documented the interpolation type as ``linear'' and the interpolation
time as one second.

\paragraph{Averaging}\label{averaging}

We document averaging methods according to five criteria: type/units,
subtype/calculation, amount, measure of center, and mean type (Figure
1). Type/units refer to the averaging units of time, breath, and digital
filters. Subtype/calculation involves specific computations like bin and
rolling averages or digital filter forms. The amount is the unit
quantity. For example, 30 for a time average is 30 seconds but is 30
breaths for a breath average. Measure of center distinguishes between
mean or median, and mean type delineates whole vs.~trimmed mean. Trimmed
(truncated) means exclude a number of the highest and lowest values in
the quantity before averaging the remaining data.

Descriptions of averaging methods are also considerably more diverse and
generic than outlier and interpolation descriptions. For example,
``30-second averages'' and ``averaged every 30 seconds'' invite
complexity, leading to more snippets referring to averaging something
besides BBB gas exchange data. Given that, we required that the text
snippets include a reference to gas data such as the text
``O\textsubscript{2},'' ``breath,'' ``gas,'' ``ventilation,'' etc.

In contrast to previous studies, we also documented every averaging
method we found per paper instead of only describing the averaging
method for VO\textsubscript{2}max. We also recorded multiple averaging
methods when the authors described the sampling interval and the
transformation applied to it. For example, the snippet from {[}34{]}

\begin{quote}
\begin{quote}
ath method using the vmax respiratory gas analyzer (sensormedics, yorba
linda, ca). vo2max was defined as the mean of the three highest values of
the averaged oxygen consumption measured consecutively OVER 20-S
intervals. a total of 98\% of the subjects achieved the respiratory
exchange ratio of ⱖ1.1. electrocardiography was recorded throughout the
exercise test using cardiosoft software (ge medical systems,
\end{quote}
\end{quote}

states that oxygen consumption was measured every 20 seconds and that
VO\textsubscript{2}max was calculated as the average of three 20-second
intervals, or 60-seconds. For this article, we documented one averaging
method as a 20-second time bin whole mean and another as a 60-second
time bin whole mean.

In many cases, authors did not explicitly use the terms ``average'' or
``mean'' to describe their averaging methods, but we documented their
methods when implied. For example, the snippet from {[}35{]} reading

\begin{quote}
\begin{quote}
red using a continuously monitored electrocardiograph. blood pressure
was measured at the end of each workload increment using an automatic
sphygmomanometer. peak v9o2 was defined as the v9o2 measured DURING THE
LAST 30 S of peak exercise. oxygen pulse was calculated by dividing v9o2
by cardiac frequency. the anaerobic threshold was detected using the
v-slope method {[}16{]}. the ventilatory equivalent for carbon dioxide w
\end{quote}
\end{quote}

states they calculated VO\textsubscript{2}peak using the last 30 seconds
of exercise data. We documented such phrasing as a 30-second time-bin
whole mean average.

\subsubsection{Data Items}\label{data-items}

In all cases, articles that did not return any phrases were documented
as ``not described'' for their respective data processing category. If
snippets did not refer to the data processing category or if the snippet
lacked sufficient information, those data processing variables were
documented as ``not described.'' For example, interpolation variables
were denoted as ``not described'' if interpolation was acknowledged but
without details for the interpolation type or time.

\paragraph{Outliers}\label{outliers-1}

We documented the outlier limit, for example, ±3 standard deviations,
and any outlier function used to compute the outlier limit, if
described.

\paragraph{Interpolation}\label{interpolation-1}

We recorded the interpolation type (linear, cubic, Lagrange,
specifically \emph{un}interpolated, and other) and time frame (e.g.,
every one second).

\paragraph{Averaging}\label{averaging-1}

We noted the following averaging types: Time, breath, breath-time,
time-breath, time-time, digital filter, ensemble, (explicitly)
\emph{un}averaged, and other. Averaging subtypes included bin, rolling,
bin-roll, rolling-bin, Butterworth low-pass, Fast Fourier Transform
(FFT), and Savitsky-Golay. Next, we recorded the time in seconds or the
number of breaths. We recorded the measure of center as mean or median.
Finally, we noted if the mean was a whole or trimmed.

\subsubsection{Synthesis of Results}\label{synthesis-of-results}

Counts, percentages, and margin of error (95\% confidence) were
calculated for the reporting frequency of each data processing method
using R {[}36{]} and RStudio {[}37{]}.

\section{Results}\label{sec-results}

\subsection{Selection of Sources of
Evidence}\label{selection-of-sources-of-evidence-1}

Figure 2 shows the selection of sources of evidence flowchart. During
our analysis, we identified 1,352 ineligible articles. We
cross-referenced those against the breath-by-breath articles and removed
another 354, leading to 8,351 articles.

\subsection{Characteristics and Results of Individual Sources of
Evidence}\label{characteristics-and-results-of-individual-sources-of-evidence}

The PRISMA Extension for Scoping Reviews checklist normally requires a
section to report the characteristics and results of individual sources
of evidence, usually in a table format, including citations {[}27{]}.
Given the vast nature of this scoping review, readers can instead view
web links to our
\href{https://docs.google.com/spreadsheets/d/1k_i4EP5U3zMltk8n21X-KHGoxUfR6XAJu6lxrVUufg0/edit?usp=sharing}{outlier},
\href{https://docs.google.com/spreadsheets/d/1mNHwyNwVeQeAAm-Jx43ImR91sLyRLSvad9oglHQB83A/edit?usp=sharing}{interpolation},
and
\href{https://docs.google.com/spreadsheets/d/1KdmDZuI1FS1XUK5zJm3JIqf4tQiBd0C1pW0p0PFZweU/edit?usp=sharing}{averaging}
data charting spreadsheets.

\subsection{Synthesis of Results}\label{synthesis-of-results-1}

We present our results according to the reporting prevalence followed by
the specific characteristics when reported.

\subsubsection{Outliers}\label{outliers-2}

Of the 8,351 articles, 376 (4.5 ± 0.4\%) reported outlier removal
methods. Of the articles reporting their outlier methods, the most
prevalent methods were ±3 (39.1\%) and ±4 (51.6\%) standard deviations,
respectively (Figure 3).

Only 102 (1.2 ± 0.2\%) articles reported details of the function they
used to calculate their outlier limit. Of those, breath-based averages
(n = 76, 74.5\%) then time-based averages (n = 15, 14.7\%) were the most
common for calculating outlier boundaries. Specifically, 5-breath
averages (n = 54, 52.9\%) were the most prevalent functions to calculate
outlier limits.

\subsubsection{Interpolation}\label{interpolation-2}

We found that 581 (7.0 ± 0.5\%) out of 8,351 specified their
interpolation methodology. When reported, the most common interpolation
time was one second (n = 527, 93.9\%). Although the majority of articles
reporting interpolation procedures did not explicitly specify their
interpolation method (n = 314, 54.0\%), linear interpolation was the
most popular stated method (n = 247, 92.5\%) (see Table 1 and Figure 4).

\begin{table}

\caption{\label{tbl-interpolation_time_type}Most prevalent specified
interpolation methods by type (a) and by time (b).}

\end{table}%

\subsubsection{Averaging}\label{averaging-2}

After removing 22 ineligible articles that we discovered during data
documentation from the original 1,100 random articles, we analyzed 1,078
articles for our averaging analysis. We recorded that 852 (66.8 ± 2.8\%)
reported some details of their data averaging methods. Time averages
dominated in popularity (91.5\%) (Table 2). Bin averages proved the most
widespread averaging subtype (89.9\%) (Table 2). Together, time-bin
(86.8\%) was the most frequent type-subtype averaging method
combination.

\begin{table}

\caption{\label{tbl-avg_type_subtype_tables}Averaging methods by type
(a) and subtype (b).}

\end{table}%

When incorporating averaging amounts, 30-, 60-, 15-, and 10-second bin
averages (Figure 5) were the most popular. The ``other'' methods
category accounted for the second highest share of the total, but this
represents many rarely used averaging methods.

\section{Discussion}\label{sec-discussion}

\subsection{Summary of Evidence}\label{summary-of-evidence}

This review shows that gas exchange data processing methods are
infrequently reported for outlier removal and interpolation. We consider
outlier removal documentation important as it applies to many exercise
test analyses. Removing outliers is important to VO\textsubscript{2}
kinetics and similar research with rapid intensity changes because they
rely on high temporal resolution. Outlier removal is also relevant for
maximal exercise testing as outliers near the end of a test may
influence VO\textsubscript{2}max or VO\textsubscript{2}peak. Previous
research indicates that a VO\textsubscript{2}max below the 20th
percentile for age and sex increases the risk of all-cause mortality
{[}38{]}, so accurate determinations of VO\textsubscript{2}max are
important for individuals with low cardiorespiratory fitness: an
erroneous breath yielding an overestimated VO\textsubscript{2}max may
subdue the urgency to improve cardiovascular health for low-fitness
individuals.

Outliers could also affect mathematical VO\textsubscript{2} plateau
determinations. Such methods test if neighboring VO\textsubscript{2}
values or a VO\textsubscript{2} vs.~time slope does not change or
increase by more than a set rate (e.g., 50 mL/min) at the end of a
maximal test. {[}9,39--42{]}. Though data averaging dampens their
influence, outliers present near the conclusion of a maximal test could
plausibly interfere with mathematical VO\textsubscript{2} plateau
determination.

We are currently unaware of research that has tested this, but outliers
may interfere with submaximal thresholds found using algorithms,
especially if they exist near likely breakpoints. Threshold algorithms
often fit piecewise linear regressions and solve for the lowest sums of
squares {[}43--45{]}. Points near the edges of the regression lines have
more leverage when solving for the best-fit line and, therefore, are
more likely to influence the slope or intercept. Such changes could
alter the intersection point of the piecewise regression, and thus, the
threshold values.

Finally, even fewer articles reported the outlier limit calculation
function. As the function chosen impacts calculated outlier limit, it
also affects where values are considered outliers. We are unaware of a
recommended outlier removal function but encourage stating such details.

We find the low interpolation reporting more reasonable because this
procedure is most relevant to less frequent VO\textsubscript{2} kinetics
studies. However, the V-slope method, one of the most common methods for
determining the first ventilatory threshold, interpolates data in their
original method {[}44{]}. Importantly, the V-slope algorithm is only
part of the overall V-slope method, so it can be unclear if authors
interpolated data when citing the V-slope method. Given this and the
artificial confidence interval shrinkage, it may be prudent for future
papers to specify interpolation or lack thereof.

Most studies use one-second linear interpolation, but different time
frames and styles, such as cubic interpolation, may yield different
results. Cubic spline interpolation produces a smooth curve but may
slightly ``overshoot'' measured values {[}46{]}. Though likely small, we
recommend authors specify the interpolation type.

Despite a much higher percentage of papers describing at least some of
their averaging methods, a third of studies examined in this review
neglected to document their process. Data averaging likely contributes
more to the final calculated values of VO\textsubscript{2}max and other
variables than do outlier removal and interpolation. Indeed, the
research on the effect of interpolation on VO\textsubscript{2} kinetics
parameters shows that interpolation does not significantly affect the
values of parameter estimates {[}20--23{]}. Although we are unaware of
studies comparing the effect of outlier removal or leaving data as-is
before proceeding with other calculations, the known impact of data
averaging on VO\textsubscript{2}max and the inherent dampening effect of
averaging on outliers itself suggests that data averaging is the most
important of the three steps when the goal is to reflect the underlying
whole-body metabolic rate. Therefore, researchers should state their gas
exchange data averaging methods to improve research reproducibility and
study comparisons.

Stating averaging methods can also help correctly classify
cardiorespiratory fitness against normative data. Research by {[}11{]}
offers a strategy to compare two VO\textsubscript{2}max values obtained
with different averaging methods. Without such corrections, one could
misclassify cardiorespiratory fitness based on VO\textsubscript{2}max if
VO\textsubscript{2}max were calculated with a sufficiently different
sampling interval than that used to generate the normative data.
Importantly, the normative data offered by the American College of
Sports Medicine {[}1, table 4.9, pp.~88-93{]} is based on a regression
of VO\textsubscript{2} vs.~time-to-exhaustion using a modified Balke
protocol and equations developed from {[}47{]} and {[}48{]} (Cooper
Institute, personal communication, 9/2021), rather than directly
measured. The system used to create the regression for males {[}48{]}
and females {[}47{]} averaged the data every minute and every 30
seconds, respectively. Given that, stating the averaging methods used
may allow for better comparisons to normative data.

The most frequent, fully specified data averaging method, the 30-second
time average, fits the maximum recommended guideline by Robergs {[}3{]}.
{[}3{]} also recommended the 15-breath rolling average or the low-pass
digital filter, but we only documented these methods two (0.2 ±0.3\%)
and one (0.1 ±0.2\%) times, respectively.

\subsection{Limitations}\label{limitations}

This study presents the most extensive review of gas exchange data
processing methods to date. However, due to its scope, not every article
received a detailed examination, which means some data processing
descriptions might have been missed due to the limitations of our
RegExs, leading us to categorize these as ``not described.'' Articles
that referred to previous works for their data processing techniques
were also marked as ``not described'' for simplicity. We realize authors
must balance adequate methodological documentation with journal word or
character limits. Yet, methodological shortcut citations can mean
missing details that prevent readers from fully reproducing the methods
used {[}49{]}. Next, by chance, we found rare examples of articles using
the median as the measure of center as we built our RegExs. However, we
did not document any such cases in our random sample. A larger sample
would likely find these and other rare data averaging methods. Finally,
it is possible that a few ineligible articles eluded our screening.
Taken together, our results are not entirely comprehensive and may
slightly underestimate data processing methods' true reporting
frequency.

Another limitation of this scoping review is that our results do not
indicate how different data processing methods were used. For example,
we did not distinguish if a 60-second time-bin average was used to
calculate VO\textsubscript{2}max or a steady-state exercise period.
Therefore, this review cannot estimate the prevalence of different
processing methods for specific analyses, such as
VO\textsubscript{2}max. Nevertheless, this is the first study we know of
to document data processing methods \emph{besides} those used to
calculate VO\textsubscript{2}max.

\subsection{Conclusions}\label{conclusions}

This scoping review found that data processing methods were seldom
reported for outlier removal and interpolation, and that averaging
reporting, though much higher, could further improve. The results
reflect prevalent methods. While prevalence should not be conflated with
quality, knowing the prevalent methods can allow others to test the
influence data processing in this field by comparing relevant options.
Finally, we hope these results motivate others to improve their
methodological documentation and, thus, reproducibility in this field.

\section*{References}\label{references}
\addcontentsline{toc}{section}{References}

\phantomsection\label{refs}
\begin{CSLReferences}{0}{1}
\bibitem[\citeproctext]{ref-pescatello2014}
\CSLLeftMargin{{[}1{]} }%
\CSLRightInline{Pescatello LS. ACSM's guidelines for exercise testing
and prescription. 9th ed. Philadelphia: Wolters Kluwer/Lippincott
Williams \& Wilkins Health; 2014}

\bibitem[\citeproctext]{ref-jamnick2020}
\CSLLeftMargin{{[}2{]} }%
\CSLRightInline{Jamnick NA, Pettitt RW, Granata C, et al. An Examination
and Critique of Current Methods to Determine Exercise Intensity. Sports
Medicine 2020; 50: 1729--1756.
doi:\href{https://doi.org/10.1007/s40279-020-01322-8}{10.1007/s40279-020-01322-8}}

\bibitem[\citeproctext]{ref-robergs2010}
\CSLLeftMargin{{[}3{]} }%
\CSLRightInline{Robergs RA, Dwyer D, Astorino T. Recommendations for
Improved Data Processing from Expired Gas Analysis Indirect Calorimetry.
Sports Medicine 2010; 40: 95--111.
doi:\href{https://doi.org/10.2165/11319670-000000000-00000}{10.2165/11319670-000000000-00000}}

\bibitem[\citeproctext]{ref-sousa2010}
\CSLLeftMargin{{[}4{]} }%
\CSLRightInline{Sousa A, Figueiredo P, Oliveira N, et al. Comparison
Between Swimming VO2peak and VO2max at Different Time Intervals. The
Open Sports Sciences Journal 2010; 3: 22--24.
doi:\href{https://doi.org/10.2174/1875399X01003010022}{10.2174/1875399X01003010022}}

\bibitem[\citeproctext]{ref-johnson1998}
\CSLLeftMargin{{[}5{]} }%
\CSLRightInline{Johnson JS, Carlson JJ, VanderLaan RL, et al. Effects of
Sampling Interval on Peak Oxygen Consumption in Patients Evaluated for
Heart Transplantation. Chest 1998; 113: 816--819.
doi:\href{https://doi.org/10.1378/chest.113.3.816}{10.1378/chest.113.3.816}}

\bibitem[\citeproctext]{ref-sell2021}
\CSLLeftMargin{{[}6{]} }%
\CSLRightInline{Sell KM, Ghigiarelli JJ, Prendergast JM, et al.
Comparison of Vȯ 2peak and Vȯ 2max at Different Sampling Intervals in
Collegiate Wrestlers. Journal of Strength and Conditioning Research
2021; 35: 2915--2917.
doi:\href{https://doi.org/10.1519/JSC.0000000000003887}{10.1519/JSC.0000000000003887}}

\bibitem[\citeproctext]{ref-midgley2007}
\CSLLeftMargin{{[}7{]} }%
\CSLRightInline{Midgley AW, McNaughton LR, Carroll S. Effect of the VȮ2
time-averaging interval on the reproducibility of VȮ2max in healthy
athletic subjects. Clinical Physiology and Functional Imaging 2007; 27:
122--125.
doi:\href{https://doi.org/10.1111/j.1475-097X.2007.00725.x}{10.1111/j.1475-097X.2007.00725.x}}

\bibitem[\citeproctext]{ref-astorino2009}
\CSLLeftMargin{{[}8{]} }%
\CSLRightInline{Astorino TA. Alterations in VO2 max and the VO2 plateau
with manipulation of sampling interval. Clinical Physiology and
Functional Imaging 2009; 29: 6067.
doi:\href{https://doi.org/10.1111/j.1475-097X.2008.00835.x}{10.1111/j.1475-097X.2008.00835.x}}

\bibitem[\citeproctext]{ref-astorino2000}
\CSLLeftMargin{{[}9{]} }%
\CSLRightInline{Astorino TA, Robergs RA, Ghiasvand F, et al. Incidence
of the oxygen plateau at VO2max during exercise testing to volitional
fatigue. Journal of exercise physiology online 2000; 3: 112}

\bibitem[\citeproctext]{ref-martin-rincon2019}
\CSLLeftMargin{{[}10{]} }%
\CSLRightInline{Martin-Rincon M, González-Henríquez JJ, Losa-Reyna J, et
al. Impact of data averaging strategies on VȮ2max assessment:
Mathematical modeling and reliability. Scandinavian Journal of Medicine
\& Science in Sports 2019; 29: 1473--1488.
doi:\href{https://doi.org/10.1111/sms.13495}{10.1111/sms.13495}}

\bibitem[\citeproctext]{ref-martin-rincon2020}
\CSLLeftMargin{{[}11{]} }%
\CSLRightInline{Martin-Rincon M, Calbet JAL. Progress Update and
Challenges on VO2max Testing and Interpretation. Frontiers in Physiology
2020; 11: 1070.
doi:\href{https://doi.org/10.3389/fphys.2020.01070}{10.3389/fphys.2020.01070}}

\bibitem[\citeproctext]{ref-scheadler2017}
\CSLLeftMargin{{[}12{]} }%
\CSLRightInline{Scheadler CM, Garver MJ, Hanson NJ. The Gas Sampling
Interval Effect on V{\textperiodcentered}O2peak Is Independent of
Exercise Protocol. Medicine \& Science in Sports \& Exercise 2017; 49:
1911--1916.
doi:\href{https://doi.org/10.1249/MSS.0000000000001301}{10.1249/MSS.0000000000001301}}

\bibitem[\citeproctext]{ref-dejesus2014}
\CSLLeftMargin{{[}13{]} }%
\CSLRightInline{Jesus K de, Guidetti L, Jesus K de, et al. Which Are The
Best VO2 Sampling Intervals to Characterize Low to Severe Swimming
Intensities? International Journal of Sports Medicine 2014; 35:
1030--1036.
doi:\href{https://doi.org/10.1055/s-0034-1368784}{10.1055/s-0034-1368784}}

\bibitem[\citeproctext]{ref-hill2003}
\CSLLeftMargin{{[}14{]} }%
\CSLRightInline{Hill DW, Stephens LP, Blumoff-Ross SA, et al. Effect of
sampling strategy on measures of VO2peak obtained using commercial
breath-by-breath systems. European Journal of Applied Physiology 2003;
89: 564--569.
doi:\href{https://doi.org/10.1007/s00421-003-0843-1}{10.1007/s00421-003-0843-1}}

\bibitem[\citeproctext]{ref-smart2015}
\CSLLeftMargin{{[}15{]} }%
\CSLRightInline{Smart NA, Jeffriess L, Giallauria F, et al. Effect of
duration of data averaging interval on reported peak VO2 in patients
with heart failure. International Journal of Cardiology 2015; 182:
530--533.
doi:\href{https://doi.org/10.1016/j.ijcard.2014.12.174}{10.1016/j.ijcard.2014.12.174}}

\bibitem[\citeproctext]{ref-matthews1987}
\CSLLeftMargin{{[}16{]} }%
\CSLRightInline{Matthews JI, Bush BA, Morales FM. Microprocessor
Exercise Physiology Systems vs a Nonautomated System. Chest 1987; 92:
696--703.
doi:\href{https://doi.org/10.1378/chest.92.4.696}{10.1378/chest.92.4.696}}

\bibitem[\citeproctext]{ref-robergs2003}
\CSLLeftMargin{{[}17{]} }%
\CSLRightInline{Robergs RA, Burnett AF. Methods Used To Process Data
From Indirect Calorimetry And Their Application To VO2 Max. Journal of
Exercise Physiology online 2003; 6}

\bibitem[\citeproctext]{ref-lamarra1987}
\CSLLeftMargin{{[}18{]} }%
\CSLRightInline{Lamarra N, Whipp BJ, Ward SA, et al. Effect of
interbreath fluctuations on characterizing exercise gas exchange
kinetics. Journal of Applied Physiology 1987; 62: 20032012.
doi:\href{https://doi.org/10.1152/jappl.1987.62.5.2003}{10.1152/jappl.1987.62.5.2003}}

\bibitem[\citeproctext]{ref-keir2014}
\CSLLeftMargin{{[}19{]} }%
\CSLRightInline{Keir DA, Murias JM, Paterson DH, et al. Breath-by-breath
pulmonary O2 uptake kinetics: Effect of data processing on confidence in
estimating model parameters. Experimental Physiology 2014; 99: 15111522.
doi:\href{https://doi.org/10.1113/expphysiol.2014.080812}{10.1113/expphysiol.2014.080812}}

\bibitem[\citeproctext]{ref-benson2017}
\CSLLeftMargin{{[}20{]} }%
\CSLRightInline{Benson AP, Bowen TS, Ferguson C, et al. Data collection,
handling, and fitting strategies to optimize accuracy and precision of
oxygen uptake kinetics estimation from breath-by-breath measurements.
Journal of Applied Physiology 2017; 123: 227242.
doi:\href{https://doi.org/10.1152/japplphysiol.00988.2016}{10.1152/japplphysiol.00988.2016}}

\bibitem[\citeproctext]{ref-francescato2014}
\CSLLeftMargin{{[}21{]} }%
\CSLRightInline{Francescato MP, Cettolo V, Bellio R. Confidence
intervals for the parameters estimated from simulated O2 uptake
kinetics: Effects of different data treatments. Experimental Physiology
2014; 99: 187195.
doi:\href{https://doi.org/10.1113/expphysiol.2013.076208}{10.1113/expphysiol.2013.076208}}

\bibitem[\citeproctext]{ref-francescato2019}
\CSLLeftMargin{{[}22{]} }%
\CSLRightInline{Francescato MP, Cettolo V. The 1-s interpolation of
breath-by-breath O2 uptake data to determine kinetic parameters: the
misleading procedure. Sport Sciences for Health 2019; 16: 193.
doi:\href{https://doi.org/10.1007/s11332-019-00602-9}{10.1007/s11332-019-00602-9}}

\bibitem[\citeproctext]{ref-francescato2015}
\CSLLeftMargin{{[}23{]} }%
\CSLRightInline{Francescato MP, Cettolo V, Bellio R. Interpreting the
confidence intervals of model parameters of breath-by-breath pulmonary
O2 uptake. Experimental Physiology 2015; 100: 475475.
doi:\href{https://doi.org/10.1113/EP085043}{10.1113/EP085043}}

\bibitem[\citeproctext]{ref-goodman2016}
\CSLLeftMargin{{[}24{]} }%
\CSLRightInline{Goodman SN, Fanelli D, Ioannidis JPA. What does research
reproducibility mean? Science Translational Medicine 2016; 8.
doi:\href{https://doi.org/10.1126/scitranslmed.aaf5027}{10.1126/scitranslmed.aaf5027}}

\bibitem[\citeproctext]{ref-opensciencecollaboration2015}
\CSLLeftMargin{{[}25{]} }%
\CSLRightInline{Open Science Collaboration. Estimating the
reproducibility of psychological science. Science 2015; 349: aac4716.
doi:\href{https://doi.org/10.1126/science.aac4716}{10.1126/science.aac4716}}

\bibitem[\citeproctext]{ref-hesse2023}
\CSLLeftMargin{{[}26{]} }%
\CSLRightInline{Hesse A.
\href{https://www.proquest.com/docview/2878185742/fulltextPDF/E53F9A0D3A0E45ADPQ/1?accountid=14586&sourcetype=Dissertations\%20&\%20Theses}{Data
Processing Methods and their Effects on the Limits of Agreement and
Reliability of Automated Submaximal Threshold Calculations}. 2023}

\bibitem[\citeproctext]{ref-tricco2018}
\CSLLeftMargin{{[}27{]} }%
\CSLRightInline{Tricco AC, Lillie E, Zarin W, et al. PRISMA Extension
for Scoping Reviews (PRISMA-ScR): Checklist and Explanation. Annals of
Internal Medicine 2018; 169: 467--473.
doi:\href{https://doi.org/10.7326/M18-0850}{10.7326/M18-0850}}

\bibitem[\citeproctext]{ref-peters2020}
\CSLLeftMargin{{[}28{]} }%
\CSLRightInline{Peters MDJ, Marnie C, Tricco AC, et al. Updated
methodological guidance for the conduct of scoping reviews. JBI Evidence
Synthesis 2020; 18: 2119--2126.
doi:\href{https://doi.org/10.11124/JBIES-20-00167}{10.11124/JBIES-20-00167}}

\bibitem[\citeproctext]{ref-ncbi2021}
\CSLLeftMargin{{[}29{]} }%
\CSLRightInline{NCBI.
\href{https://www.ncbi.nlm.nih.gov/pmc/tools/id-converter-api/}{ID
Converter API - PMC}. 2021;}

\bibitem[\citeproctext]{ref-bojanowski2016}
\CSLLeftMargin{{[}30{]} }%
\CSLRightInline{Bojanowski P, Grave E, Joulin A, et al. Enriching Word
Vectors with Subword Information. arXiv preprint arXiv:160704606 2016;}

\bibitem[\citeproctext]{ref-pedregosa2011}
\CSLLeftMargin{{[}31{]} }%
\CSLRightInline{Pedregosa F, Varoquaux G, Gramfort A, et al.
Scikit-learn: Machine Learning in Python. Journal of Machine Learning
Research 2011; 12: 28252830}

\bibitem[\citeproctext]{ref-breese2019}
\CSLLeftMargin{{[}32{]} }%
\CSLRightInline{Breese BC, Saynor ZL, Barker AR, et al. Relationship
between (non)linear phase II pulmonary oxygen uptake kinetics with
skeletal muscle oxygenation and age in 11{\textendash}15 year olds.
Experimental Physiology 2019; 104: 1929--1941.
doi:\href{https://doi.org/10.1113/EP087979}{10.1113/EP087979}}

\bibitem[\citeproctext]{ref-hartman2018}
\CSLLeftMargin{{[}33{]} }%
\CSLRightInline{Hartman ME, Ekkekakis P, Dicks ND, et al. Dynamics of
pleasure-displeasure at the limit of exercise tolerance: conceptualizing
the sense of exertional physical fatigue as an affective response.
Journal of Experimental Biology 2018; jeb.186585.
doi:\href{https://doi.org/10.1242/jeb.186585}{10.1242/jeb.186585}}

\bibitem[\citeproctext]{ref-hassinen2008}
\CSLLeftMargin{{[}34{]} }%
\CSLRightInline{Hassinen M, Lakka TA, Savonen K, et al.
Cardiorespiratory Fitness as a Feature of Metabolic Syndrome in Older
Men and Women. Diabetes Care 2008; 31: 1242--1247.
doi:\href{https://doi.org/10.2337/dc07-2298}{10.2337/dc07-2298}}

\bibitem[\citeproctext]{ref-deboeck2004}
\CSLLeftMargin{{[}35{]} }%
\CSLRightInline{Deboeck G, Niset G, Lamotte M, et al. Exercise testing
in pulmonary arterial hypertension and in chronic heart failure.
European Respiratory Journal 2004; 23: 747--751.
doi:\href{https://doi.org/10.1183/09031936.04.00111904}{10.1183/09031936.04.00111904}}

\bibitem[\citeproctext]{ref-rcoreteam2021}
\CSLLeftMargin{{[}36{]} }%
\CSLRightInline{R Core Team. \href{https://www.R-project.org/}{R: A
Language and Environment for Statistical Computing}. Vienna, Austria: R
Foundation for Statistical Computing; 2021}

\bibitem[\citeproctext]{ref-positteam2022}
\CSLLeftMargin{{[}37{]} }%
\CSLRightInline{Posit team. \href{http://www.posit.co/}{RStudio:
Integrated Development Environment for R}. Boston, MA: Posit Software,
PBC; 2022}

\bibitem[\citeproctext]{ref-blair1995}
\CSLLeftMargin{{[}38{]} }%
\CSLRightInline{Blair SN, Kohl HW, Barlow CE, et al. Changes in physical
fitness and all-cause mortality: a prospective study of healthy and
unhealthy men. Jama 1995; 273: 10931098.
doi:\href{https://doi.org/10.1001/jama.1995.03520380029031}{10.1001/jama.1995.03520380029031}}

\bibitem[\citeproctext]{ref-robergs2001}
\CSLLeftMargin{{[}39{]} }%
\CSLRightInline{Robergs RA. An exercise physiologist's
{"}contemporary{"} interpretations of the{"} ugly and creaking
edifices{"} of the VO2max concept. Journal of Exercise Physiology Online
2001; 4: 144}

\bibitem[\citeproctext]{ref-myers1989}
\CSLLeftMargin{{[}40{]} }%
\CSLRightInline{Myers J, Walsh D, Buchanan N, et al. Can Maximal
Cardiopulmonary Capacity be Recognized by a Plateau in Oxygen Uptake?
Chest 1989; 96: 1312--1316.
doi:\href{https://doi.org/10.1378/chest.96.6.1312}{10.1378/chest.96.6.1312}}

\bibitem[\citeproctext]{ref-myers1990}
\CSLLeftMargin{{[}41{]} }%
\CSLRightInline{Myers J, Walsh D, Sullivan M, et al. Effect of sampling
on variability and plateau in oxygen uptake. Journal of Applied
Physiology 1990; 68: 404--410.
doi:\href{https://doi.org/10.1152/jappl.1990.68.1.404}{10.1152/jappl.1990.68.1.404}}

\bibitem[\citeproctext]{ref-yoon2007}
\CSLLeftMargin{{[}42{]} }%
\CSLRightInline{Yoon B-K, Kravitz L, Robergs R. V O2max, protocol
duration, and the V O2 plateau. Medicine \& Science in Sports \&
Exercise 2007; 39: 11861192}

\bibitem[\citeproctext]{ref-jones1984}
\CSLLeftMargin{{[}43{]} }%
\CSLRightInline{Jones RH, Molitoris BA. A statistical method for
determining the breakpoint of two lines. Analytical Biochemistry 1984;
141: 287--290.
doi:\href{https://doi.org/10.1016/0003-2697(84)90458-5}{10.1016/0003-2697(84)90458-5}}

\bibitem[\citeproctext]{ref-beaver1986}
\CSLLeftMargin{{[}44{]} }%
\CSLRightInline{Beaver WL, Wasserman K, Whipp BJ. A new method for
detecting anaerobic threshold by gas exchange. Journal of Applied
Physiology 1986; 60: 2020--2027.
doi:\href{https://doi.org/10.1152/jappl.1986.60.6.2020}{10.1152/jappl.1986.60.6.2020}}

\bibitem[\citeproctext]{ref-orr1982}
\CSLLeftMargin{{[}45{]} }%
\CSLRightInline{Orr GW, Green HJ, Hughson RL, et al. A computer linear
regression model to determine ventilatory anaerobic threshold. Journal
of Applied Physiology 1982; 52: 1349--1352.
doi:\href{https://doi.org/10.1152/jappl.1982.52.5.1349}{10.1152/jappl.1982.52.5.1349}}

\bibitem[\citeproctext]{ref-zhang1997}
\CSLLeftMargin{{[}46{]} }%
\CSLRightInline{Zhang Z, Martin CF. Convergence and Gibbs' phenomenon in
cubic spline interpolation of discontinuous functions. Journal of
Computational and Applied Mathematics 1997; 87: 359--371.
doi:\href{https://doi.org/10.1016/S0377-0427(97)00199-4}{10.1016/S0377-0427(97)00199-4}}

\bibitem[\citeproctext]{ref-pollock1982}
\CSLLeftMargin{{[}47{]} }%
\CSLRightInline{Pollock ML, Foster C, Schmidt D, et al. Comparative
analysis of physiologic responses to three different maximal graded
exercise test protocols in healthy women. American heart journal 1982;
103: 363373}

\bibitem[\citeproctext]{ref-pollock1976}
\CSLLeftMargin{{[}48{]} }%
\CSLRightInline{Pollock ML, Bohannon RL, Cooper KH, et al. A comparative
analysis of four protocols for maximal treadmill stress testing.
American Heart Journal 1976; 92: 39--46.
doi:\href{https://doi.org/10.1016/S0002-8703(76)80401-2}{10.1016/S0002-8703(76)80401-2}}

\bibitem[\citeproctext]{ref-standvoss2022}
\CSLLeftMargin{{[}49{]} }%
\CSLRightInline{Standvoss K, Kazezian V, Lewke BR, et al.
\href{https://doi.org/10.1101/2022.08.08.503174}{Taking shortcuts: Great
for travel, but not for reproducible methods sections}. 2022}

\end{CSLReferences}



\end{document}
